\section{Metric Space}

We spend this chapter introducing a metric space and building some intuition around it.
From the definition below, if we can show they are true, then we have a metric space.

\begin{definition}
  A \textbf{metric space} is a pair $(X, d)$ where $X$ is a set, and $d$ is a \textit{metric on $X$} (or \textit{distance function on $X$}), that is, a function defined on $X \times X$ such that for all $x, y, z \in X$ we have:
  \begin{enumerate}
    \item $d$ is real-valued, finite and nonnegative
    \item $d(x, y) = 0$ if and only if $x = y$
    \item $d(x, y) = d(y, x)$ (Symmetry)
    \item $d(x, y) \leq d(x, z) + d(z, y)$ (\textbf{Triangle inequality})
  \end{enumerate}
  \label{def:metric_space}
\end{definition}

\bx{
  \label{chap1-1:ex1}
  Let's show that $\mathbb{R}$ is a metric space. Since the author didn't specify, I assume they mean the standard metric $d(x, y) = |x - y|$.

  In this case, we can show
  \begin{enumerate}
    \item $d(x, y) = |x - y| \geq 0$, by definition of absolute value. It is finite, since $x, y$ are finite, and you can bound $|x - y|$ by $|x| + |y|$ which is also finite. And it is $\in \mathbb{R}$ by definition of absolute value.
    \item If $x=y$ then $d(x, y) = |x - y| = 0$. If $d(x, y) = 0$, then $|x - y| = 0$, which means $x - y = 0$, or $x = y$.
    \item $d(x, y) = |x - y| = |y - x| = d(y, x)$.
    \item $d(x, y) = |x - y| = |x - z + z - y| \leq |x - z| + |z - y| = d(x, z) + d(z, y)$
  \end{enumerate}
}

\bx{
  Same as \ref{chap1-1:ex1}, but for $d(x, y) = (x - y)^2$.

  Counterexample for Triangle Inequality: Let $x = 0, y = 2, z = 1$.
  Then, we have
  \begin{align*}
    d(x, y) &= (2-0)^2 = 4\\
    d(x, z) + d(z, y) &= (0-1)^2 + (1-2)^2 = 1 + 1 = 2 \\
    4 &\not\leq 2
  \end{align*}

  So it is \textbf{not a metric}.
}

\bx{
  Same as \ref{chap1-1:ex1}, but for $d(x, y) = \sqrt{|x - y|}$.

  We can show
  \begin{enumerate}
    \item $d(x, y) = \sqrt{|x - y|} \geq 0$, by definition of absolute value and square root. It is finite, since $x, y$ are finite, and you can bound $\sqrt{|x - y|}$ by $\sqrt{|x| + |y|}$ which is also finite. And it is $\in \mathbb{R}$ by definition of absolute value and square root.
    \item If $x=y$ then $d(x, y) = \sqrt{|x - y|} = 0$. If $d(x, y) = 0$, then $\sqrt{|x - y|} = 0$, which means $|x - y| = 0$, or $x - y = 0$, or $x = y$.
    \item $d(x, y) = \sqrt{|x - y|} = \sqrt{|y - x|} = d(y, x)$.
    \item $d(x, y) = \sqrt{|x - y|} = \sqrt{|x - z + z - y|} \leq \sqrt{|x - z|} + \sqrt{|z - y|} = d(x, z) + d(z, y)$\\
      The inequality step comes from the fact that $\sqrt{a + b} \leq \sqrt{a} + \sqrt{b}$ for nonnegative $a, b$, because
      \begin{align*}
        (\sqrt{a} + \sqrt{b})^2 &\,?\, (\sqrt{a+b})^2\\
        a + b + 2\sqrt{ab} &\geq a + b
      \end{align*}
  \end{enumerate}

  So it is a metric.
}

\bx{
  We need to find all metrics on a set $X$ consisting of two points.

  Let $X = \{a, b\}$ with $a \neq b$. Suppose we have the following metric:

  \[
    d(x,y) =
    \begin{cases}
      0 & x = y \\
      c & x \neq y , c > 0
    \end{cases}
  \]

  We can verify 1, 2, 3, from \ref{def:metric_space} very easily. The main caveat, we needed $c>0$ for 2, for $d(x, y) = 0$ iff $x = y$.
  Then, for 4 (Triangle Inequality), we can check all possible cases:
  \begin{enumerate}
    \item If $x = y$, then $d(x, y) = 0 \leq d(x, z) + d(z, y)$
    \item if $x \neq y$, then we have two more cases
      \begin{enumerate}
        \item if $x = z$, then $z\neq y$ and we have $d(x, z) + d(z, y) = c \geq c = d(x, y)$
        \item if $x \neq z$, then we have $z = y$ and similarly $d(x, z) + d(z, y) = c \geq c = d(x, y)$
      \end{enumerate}
  \end{enumerate}

  If $X$ consists of one point, then the only metric is the trivial one where $d(x, x) = 0$.
  This is because, if we have any other metric, then $d(x, x) > 0$ which violates property 2.
}

\bx{
  Suppose $d$ is a metric on $X$.

  \begin{enumerate}
    \item Suppose we have $kd$. Then this is still a metric if $k > 0$.
      We can verify the four properties:
      \begin{enumerate}
        \item $kd(x, y) \geq 0$ since $d(x, y) \geq 0$ and $k > 0$. It is finite since $d$ is finite and $k$ is a finite scalar.
        \item $kd(x, y) = 0$ iff $d(x, y) = 0$ iff $x = y$.
        \item $kd(x, y) = k d(y, x)$, we can just scale property 3
        \item $kd(x, y) \leq kd(x, z) + kd(z, y)$ we can just scale property 4 the triangle inequality.
      \end{enumerate}
    \item Suppose we have $d + k$. Then this is only a metric for $k=0$. If $k\neq 0$ then property 2 is immediately dissatisfied, since $d(x, x) + k = k \neq 0$.
  \end{enumerate}
}

\bx{
  We want to show the sequnce space $l^\infty$ is a metric space.

  \begin{enumerate}
    \item $d(x, y) = \sup_{n \in \mathbb{N}} |x_n - y_n| \geq 0$ since absolute value is nonnegative. It is finite since $x, y \in l^\infty$ means they are bounded sequences by $c_x \in \mathbb{R}$, so their difference is also bounded. And it is $\in \mathbb{R}$ since absolute value produces real numbers.
    \item If $x = y$, then $d(x, y) = \sup |x_n - y_n| = 0$. If $d(x, y) = 0$, then $\sup |x_n - y_n| = 0$, which means for all $n$, $|x_n - y_n| = 0$, or $x_n = y_n$ for all $n$, so $x = y$.
    \item $d(x, y) = \sup |x_n - y_n| = \sup |y_n - x_n| = d(y, x)$.
    \item $d(x, y) = \sup |x_n - y_n| = \sup |x_n - z_n + z_n - y_n| \leq \sup |x_n - z_n| + \sup |z_n - y_n| = d(x, z) + d(z, y)$.
  \end{enumerate}
}

\bx{
  Supposed $A$ is the subset of $l^\infty$ where each sequence is composed of 0 or 1.

  The induced metric then behaves like this:

  \[
    d(x, y) =
    \begin{cases}
      0 & x = y \\
      1 & x \neq y
    \end{cases}
  \]
}

\bx{
  We have $X$ as the set of all real-valued functions, and are defined and continuous on a closed interval $J = \bracken{a, b}$.
  We now choose the metric
  \begin{equation}
    d(x, y) = \int_a^b |x(t) - y(t)| \dd{t}
  \end{equation}

  Let's check this is a metric
  \begin{enumerate}
    \item $d(x, y) = \int_a^b |x(t) - y(t)| \dd{t} \geq 0$ since absolute value is nonnegative. It is finite since $x, y$ are continuous on a closed interval, so they are bounded, and the integral of a bounded function over a finite interval is finite. And it is $\in \mathbb{R}$ since the integral of a real-valued function is real.
    \item If $x = y$, then $d(x, y) = \int_a^b |x(t) - y(t)| \dd{t} = \int_a^b |0| \dd{t} = 0$. If $d(x, y) = 0$, then $\int_a^b |x(t) - y(t)| \dd{t} = 0$, which implies $|x(t) - y(t)| = 0$ for all $t \in [a,b]$, so $x(t) = y(t)$ for all $t \in [a,b]$.
    \item $d(x, y) = \int_a^b |x(t) - y(t)| \dd{t} = \int_a^b |y(t) - x(t)| \dd{t} = d(y, x)$.
    \item $d(x, y) = \int_a^b |x(t) - y(t)| \dd{t} = \int_a^b |x(t) - z(t) + z(t) - y(t)| \dd{t} \leq \int_a^b |x(t) - z(t)| \dd{t} + \int_a^b |z(t) - y(t)| \dd{t} = d(x, z) + d(z, y)$.
  \end{enumerate}
}

\bx{
  We want to show that the discrete metric for $X$ is a metric, this is
  \begin{equation}
    d(x, x) = 0, d(x, y) = 1 \text{ for } x \neq y
  \end{equation}

  \begin{enumerate}
    \item $d(x, y)$ is either 0 or 1, so it is nonnegative, finite, and real-valued.
    \item By definition, $d(x, y) = 0$ iff $x = y$.
    \item By definition, $d(x, y) = d(y, x)$.
    \item For the triangle inequality, we have two cases:
      \begin{enumerate}
        \item If $x = y$, then $d(x, y) = 0 \leq d(x, z) + d(z, y)$
        \item If $x \neq y$, then we have two more cases
          \begin{enumerate}
            \item if $x = z$, then $z\neq y$ and we have $d(x, z) + d(z, y) = 1 \geq 1 = d(x, y)$
            \item if $x \neq z$, then we have $z = y$ and similarly $d(x, z) + d(z, y) = 1 \geq 1 = d(x, y)$
          \end{enumerate}
      \end{enumerate}
  \end{enumerate}
}

\bx{
  \textbf{(Hamming distance)} Let $X$ be all set of ordered triples of 0s and 1s.

  We immediately see that $X$ has $2^3 = 8$ elements.

  Now, $d(x, y) = \text{number of positions where } x \text{ and } y \text{ differ}$.

  \begin{enumerate}
    \item $d(x, y) = 0, 1, 2, 3$ so it is nonnegative, finite, and real-valued.
    \item If $x = y$, then they differ in 0 positions, so $d(x, y) = 0$. If $d(x, y) = 0$, then they differ in 0 positions, so $x$ and $y$ must be the same in all positions, so $x = y$.
    \item The number of positions where $x$ and $y$ differ is the same as the number of positions where $y$ and $x$ differ, so $d(x, y) = d(y, x)$.
    \item For the triangle inequality, we can prove this my induction using the discrete metric space, as the base case.
      Then, suppose we have $n$-tuples, and we add one more position to each tuple to make it an $(n+1)$-tuple.
      There are two cases:
      \begin{enumerate}
        \item If the new position is the same for all three tuples, then the number of differing positions is unchanged, so the triangle inequality holds by the inductive hypothesis.
        \item If the new position is different for at least one of the tuples, then the left-hand side of the triangle inequality increases by at most 1, and the right-hand side also increases by at least 1, because in $z$, its new position must be different from at least $x$ or $y$.
      \end{enumerate}
  \end{enumerate}
}

\bx{
  We want to prove the generalized triangle inequality:
  \begin{equation}
    d(x_1, x_n) \leq d(x_1, x_2) + d(x_2, x_3) + \dots + d(x_{n-1}, x_n)
  \end{equation}
  We can do this by induction.

  Base case: $n=2$ is just the regular triangle inequality.
  Inductive step: Suppose it is true for $n=k$.
  Then, for $n = k+1$, we have
  \begin{align*}
    d(x_1, x_{k+1}) &\leq d(x_1, x_k) + d(x_k, x_{k+1}) \tag{Triangle Inequality} \\
    &\leq d(x_1, x_2) + d(x_2, x_3) + \dots + d(x_{k-1}, x_k) + d(x_k, x_{k+1}) \tag{Inductive Hypothesis}
  \end{align*}
  Thus, by induction, the generalized triangle inequality holds for all $n \geq 2$.
}

\bx{
  We can see that
  \begin{align*}
    \abs{d(x, y) - d(z, w)} &\leq d(x, z) + d(y, w) \\
    d(x, y) - d(z, w) &\leq d(x, z) + d(y, w) \tag{Case 1, we'll do the other case next}\\
    d(x, y) &\leq d(x, z) + d(y, w) + d(z, w) \\
    d(x, y) &\leq d(x, z) + d(w, y) + d(z, w) \tag{Symmetry}\\
    d(x, y) &\leq d(x, z) + d(z, w) + d(w, y) \tag{Triangle Inequality}
  \end{align*}

  We need to do the other case as well:
  \begin{align*}
    d(z, w) - d(x, y) &\leq d(x, z) + d(y, w) \\
    d(z, w) &\leq d(x, z) + d(y, w) + d(x, y) \\
    d(z, w) &\leq d(z, x) + d(x, y) + d(y, w) \tag{Symmetry}\\
    d(z, w) &\leq d(z, x) + d(x, y) + d(y, w) \tag{Triangle Inequality}
  \end{align*}
}

\bx{
  \begin{align*}
    \abs{d(x, z) - d(y, z)} &\leq d(x, y) \\
    d(x, z) - d(y, z) &\leq d(x, y) \tag{Case 1, we'll do the other case next}\\
    d(x, z) &\leq d(x, y) + d(y, z) \tag{Triangle Inequality}\\
  \end{align*}

  Let's do the other case:
  \begin{align*}
    d(y, z) - d(x, z) &\leq d(x, y) \\
    d(y, z) &\leq d(x, y) + d(x, z) \\
    d(y, z) &\leq d(y, x) + d(x, z) \tag{Symmetry}\\
    d(y, z) &\leq d(y, x) + d(x, z) \tag{Triangle Inequality}
  \end{align*}
}

\bx{
  We want to show that properties 2-4 in the \ref{def:metric_space} definition imply that the metric must be nonnegative.

  AFSOC that there exists some $x, y \in X$ such that $d(x, y) < 0$.
  Then, by property 4, we have
  \begin{equation*}
    d(x, x) \leq d(x, y) + d(y, x)
  \end{equation*}
  By property 3, we have $d(y, x) = d(x, y)$, so
  \begin{equation*}
    d(x, x) \leq 2d(x, y)
  \end{equation*}
  But by property 2, we have $d(x, x) = 0$, so
  \begin{equation*}
    0 \leq 2d(x, y)
  \end{equation*}
  However, we assumed $d(x, y) < 0$, so $2d(x, y) < 0$, which is a contradiction.
  Therefore, our assumption is false, and we conclude that for all $x, y \in X$, $d(x, y) \geq 0$.
}